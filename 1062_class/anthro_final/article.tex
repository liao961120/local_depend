\documentclass[]{tufte-handout}

% ams
\usepackage{amssymb,amsmath}

\usepackage{ifxetex,ifluatex}
\usepackage{fixltx2e} % provides \textsubscript
\ifnum 0\ifxetex 1\fi\ifluatex 1\fi=0 % if pdftex
  \usepackage[T1]{fontenc}
  \usepackage[utf8]{inputenc}
\else % if luatex or xelatex
  \makeatletter
  \@ifpackageloaded{fontspec}{}{\usepackage{fontspec}}
  \makeatother
  \defaultfontfeatures{Ligatures=TeX,Scale=MatchLowercase}
  \makeatletter
  \@ifpackageloaded{soul}{
     \renewcommand\allcapsspacing[1]{{\addfontfeature{LetterSpace=15}#1}}
     \renewcommand\smallcapsspacing[1]{{\addfontfeature{LetterSpace=10}#1}}
   }{}
  \makeatother

\fi

% graphix
\usepackage{graphicx}
\setkeys{Gin}{width=\linewidth,totalheight=\textheight,keepaspectratio}

% booktabs
\usepackage{booktabs}

% url
\usepackage{url}

% hyperref
\usepackage{hyperref}

% units.
\usepackage{units}


\setcounter{secnumdepth}{-1}

% citations

% pandoc syntax highlighting

% longtable
\usepackage{longtable,booktabs}

% multiplecol
\usepackage{multicol}

% strikeout
\usepackage[normalem]{ulem}

% morefloats
\usepackage{morefloats}


% tightlist macro required by pandoc >= 1.14
\providecommand{\tightlist}{%
  \setlength{\itemsep}{0pt}\setlength{\parskip}{0pt}}

% title / author / date
\title{開源社群:}
\author{廖永賦}
\date{2018-07-01}

%加這個就可以設定字體
\usepackage{fontspec}
%使用xeCJK,其他的還有CJK或是xCJK
\usepackage{xeCJK}
\usepackage{bm}

% Pandoc 只能設置 10, 11, 12 pt 超出此範圍需用此指令
\usepackage[fontsize=12pt]{scrextend}

%設定英文字型,不設的話就會使用預設的字型
\setmainfont{Calibri}

%設定中英文的字型
%字型的設定可以使用系統內的字型,而不用像以前一樣另外安裝
% AR PL KaitiM Big5 標楷體
% .PingFang TC
% Noto Sans CJK TC
% HanWangHeiLight
% HanWangHeiHeavy
\setCJKmainfont[
	BoldFont={HanWangHeiHeavy}  % 粗體字使用華康粗明體
    ]{AR PL KaitiM Big5}    % 一般字型則使用華康細明體

%中文自動換行
\XeTeXlinebreaklocale "zh"

%文字的彈性間距
\XeTeXlinebreakskip = 0pt plus 1pt

%設定段落之間的距離
\setlength{\parskip}{0.15cm}

%設定行距
\linespread{1.1}

\begin{document}

\maketitle




\subsection{摘要}

\subsection{緒論}

自由與開源軟體(``Free and open-source software,''
\protect\hyperlink{ref-2018h}{2018})不同於版權軟體(如微軟的
Office、Google 的 Gmail 等),是開放原始碼\footnote{軟體的原始碼是由高階的程式語言所撰寫,如常見的
  C++ 語言、Java
  等,這些語言是人類能夠「看懂」的語言。一個軟體要能在電腦上運行,需要將高階的程式語言轉換成低階的機械碼。機械碼是電腦「看的懂」的語言(全部由
  0 或 1
  組成),但人類很難讀懂的語言,且機械碼也無法被轉換回高階語言。因此,私有的版權軟體為保護自己的智慧財產,通常僅以機械碼的形式將軟體發布給使用者。開源軟體則會公開其原始碼,讓使用者能夠修改該軟體。}的軟體。開放原始碼意味使用者能自由的使用軟體,例如,根據自己的喜好修改軟體以符合使用的需求或偏好,並且也能向軟體開發者提供修改軟體的建議或程式碼。開源軟體因而能吸收來自網路上眾多「駭客」\footnote{駭客(hacker)並非大眾媒體上所指「破解電腦以竊取資料」的電腦高手,其正確的詞彙應為
  `cracker'(Raymond,
  n.d.)。駭客在開源社群或更廣泛的駭客社群中,是帶有正面意義與認同的詞彙。}的貢獻,增加其修改軟體缺陷的速度與能力(Raymond,
\protect\hyperlink{ref-raymond1999}{1999}),而產生許多優秀的開源軟體。

一般的版權軟體不會

\subsection{歷史背景}

\subsection{開源社群的運作}

\subsubsection{個人動機}

\subsubsection{禮物交換}

\subsection*{參考資料}\label{reference}
\addcontentsline{toc}{subsection}{參考資料}

\hypertarget{refs}{}
\hypertarget{ref-2018h}{}
Free and open-source software. (2018). \emph{Wikipedia}.

\hypertarget{ref-raymond1999}{}
Raymond, E. (1999). The cathedral and the bazaar. \emph{Knowledge,
Technology \& Policy}, \emph{12}(3), 23--49.
\url{https://doi.org/10.1007/s12130-999-1026-0}

\hypertarget{ref-raymond}{}
Raymond, E. (n.d.). The Jargon File: Hacker.
http://www.catb.org/jargon/html/H/hacker.html.



\end{document}
